\section{samsung.com/no}

\subsection{Målgruppe}

Målgruppen for Samsung er ganske stor og i hovedsak veldig lik, hvis ikke den samme som Apple. I tillegg har Samsung også en del andre elektronikkprodukter som harddisker, hjemmekino etc\ldots

\subsection{Visuell framstilling}

\subsubsection{Hvordan er det visuelle uttrykket tilpasset målgruppen}

Samsung går for et mer komplekst design enn Apple. De har også gruppert designet godt med klare, definerte knapper. Designet er mer krevende og vil for bestemoren trolig virke mer krevende, ihvertfall før hun får satt seg inn i det. Det er skarpe farger med mye kontrast for å skille komponenter i designet. 

\subsubsection{Brukes gestaltprinsippene}

\begin{description}
  \item[Gruppering/Nærhet] \hfill \\
  Alle lignende elementer er samlet i grupper som produkter, bedrift, support. Det er også en egen gruppe for mobil, foto, tv etc. 
  \item[Linje/Kontinuitet] \hfill \\
  Vil si nettsiden stiller sterkt her, da det meste er organisert i horistontale grupper med klare skiller.
  \item[Mental komplettering] \hfill \\
    Kan ikke si jeg ser noe konkret eksempel på dette
   \item[Likhet i form] \hfill \\
    Det meste på nettsiden er veldig firkantet og likt i form. 
  \item[Likhet i farge] \hfill \\
    Det er tydlig at det er benyttet en fargepallett som er gjennomgående for siden.
  \item[Forgrunn/Bakgrunn + reversering] \hfill \\
    Hovedmenyen er fremhevet noe i forhold til reklamene/annonsene som ruller under.
\end{description}

\subsubsection{Jacob Nilesens liste}

\begin{description}
  \item[1. Visibility of system status] \hfill \\
    Det er begrenset hvor mye status en bruker kan se på en nettside, men menylinja øverst viser til enhver tid hvilken fane brukeren er i.
  \item[2. Match between system and real world] \hfill \\
    Det brukes jevnt over et naturlig språk, som er lett å forstå. Dette er sett bort i fra eventuelle tekniske begreper, men disse blir ofte forklart i liten skrift. 
  \item[3. User control and freedom] \hfill \\
    Brukeren kan når som helst bevege seg dit han eller hun ønsker. 
  \item[4. Consistency and standards] \hfill \\
    Ser ingen tilfeller hvor dette ikke oppfylles. 
  \item[5. Error prevention] \hfill \\
   Ser ikke relevansen for denne. 
  \item[6. Recognintion reather than recall] \hfill \\
  Synes nettsiden gjør en god jobb med dette. Det er lett å forstå hva de ulike linkene/knappene gjør.
  \item[7. Flexibility and effiency of use] \hfill \\
    Det er ikke noe særlig i nettsiden som øker effektiviteten for en ekspertbruker, men det er søkefelt tilgjengelig som kan øke effektiviteten litt. 
  \item[8. Aestetic and minimalist design] \hfill \\
    Det er et litt mer komplekst design enn hos Apple, med mer tekst og knapper på fremsiden. Det kan ikke sies at dette er et godt minimalistisk design
  \item[9. Help users recognize, diagnose, and recover from errors] \hfill \\
    Lite relevant for en nettbutikk/internettside  
  \item[10. Help and documentation] \hfill \\
   Lett å nå supportside, med knapp både oppe og nede på nettsiden. 
\end{description}

\subsubsection{Schneiderman}

\begin{description}
  \item[Strive for Consistency] \hfill \\
    Synes siden er veldig konsitent i designvalg. 
  \item[Cater of universal usability] \hfill \\
    Finner ingen tastaturkommandoer for snarveier.
  \item[Offer informative feedback] \hfill \\
    Linker i menyen endrer farge når musa panorerer over. Denne endringen av utseende gir en god feedback. 
  \item[Design dialogs to yield closure] \hfill \\
    Har ikke forsøkt et kjøp i nettbutikken, så kan ikke svare godt på dette spørsmålet.
  \item[Prevent errors] \hfill \\
    Vet ikke hvordan dette løses av nettsiden
  \item[Permit easy reversal of actions] \hfill \\
    Samme som forrige 
  \item[Support internal locus of control] \hfill \\
    Er ingen mulighet for å modifisere siden som man vil eller å logge inn på et personlig ``dashboard''. 
  \item[Reduce short term memory] \hfill \\
    Det er lett å manuvrere frem og tilbake, så dette skulle ikke være et problem. 
\end{description}

\subsection{Delt opp etter brukergrupper?}

Forsiden til Samsung er delt opp slik at du kan velge overordnet tema for hva du er ute etter i toppmenyen. For eksempel ``Produkter'' og ``Bedrift''. Dette er forskjellig fra Apples løsning, hvor bare produktet kan velges, ikke brukergruppe 

\subsection{Hensikten med nettstedet}

Typisk brukssituasjon: 

\begin{itemize}
\item Finne en ny flatskjerm TV
\item Se på de nyeste telefonene
\end{itemize}

\subsubsection{Bestemoren vil ha ny TV}

Bestemor Berit er lei av sin gamle kassetv og ønsker å bli med på den digitale TV-revolusjonen. Hun har derfor benyttet et av datakursene til å komme seg inn på samsung.no. Hun finner fort frem til ``TV \& Hjemmekino''-fanen hvor hun for oversikt over alle Samsungs TV-typer. Hun klikker videre på LED-TV fordi barnebarnet har anbefalt denne typen. Her får hun se mange forskjellige TVer.  

\subsubsection{Familiefaren vurderer ny telefon}

Øivind er lei av det dårlige batteriet på telefonen sin. Han går derfor inn på Samsung-nettsiden for å sjekke ut det nyeste innen Androidtelefoner. Han klikker seg inn på telefoner og smartphones. Han får opp alle samsungs telefoner og kan sortere de på flere måter. Han finner til slutt en Galaxy S4 han ser ut til å like. Han bestemmer seg derfor for å dra i butikken og kjøpe denne telefonen. 

\subsection{Språk/ordvalg}

I likhet med Apple har Samsung benyttet seg av et nøye gjennomtenkt ordvalg. Det er benyttet korte fraser for å vekke interesse om produktet. Ønsker man mer informasjon må man trykke seg inn på produktene/annonsene.  

\subsection{Linker, knapper, menyer\dots}

Selve menybaren er ganske lettforsåelig og det er tyldig hva som er knapper når man drar musa over. Nyheter kommer tydlig frem i rammen som er nedsenket iforhold til menyene. Hover-effekten på menybaren gjør at produktgrupper og underknapper kommer frem når musa panorerer over. 


\subsection{Totalvurdering av brukervennlighet}

Jeg vil si at denne siden har en veldig brukervennlig design, men kan ha noe forstyrrende fargevalg til tider. Den ivaretar mange designprinsipper.  Den er veldig brukervennlig for de fleste målgrupper. Det er bra med gode fargekontraster, men noen steder på siden føler jeg kontrastene blir mer forstyrrende enn brukervennlig. Det kan også diskuteres om det under nyhetsgruppa er for mange grupper som virker forstyrrende. 