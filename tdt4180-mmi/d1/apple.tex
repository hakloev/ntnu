\section{apple.com/no}

\subsection{Målgruppe}

Målgruppen for Apple er ganske stor. I hovedsak vil man kunne si at målgruppen for Apple er alle som ønsker å gå til anskaffelse av data, mobil eller musikkspiller. I tillegg selger nettbutikken også tilbehør til alle disse enhetene. Det skal legges til at de sammen med Samsung har en av de ledende posisjonene på markedet og derfor trolig når ut til nesten hele markedet. 

\subsection{Visuell framstilling}

\subsubsection{Hvordan er det visuelle uttrykket tilpasset målgruppen}

Apple går for en minimalistisk design, med noen klare farger. Til tider dukker det opp endel farger som skaper tydlig kontrast. 

\subsubsection{Brukes gestaltprinsippene}

\begin{description}
  \item[Gruppering/Nærhet] \hfill \\
  Alle lignende elementer er samlet i grupper, som meny, informasjon, nyheter.
  \item[Linje/Kontinuitet] \hfill \\
  Vil si nettsiden stiller sterkt her, da det meste er organisert i linjer/grupper med rette vinkler.
  \item[Mental komplettering] \hfill \\
    Kan ikke si jeg ser noe konkret eksempel på dette
   \item[Likhet i form] \hfill \\
  Bortsett fra den øverste menylinjen er forsiden preget av likhet i form. Det skal legges til at det er  litt runde hjørner på menylinjen, noe som er ulikt resten
  \item[Likhet i farge] \hfill \\
  Det går mye i hvitt og grått, men det dukker til tider opp sterke farger som skaper kontrast.
  \item[Forgrunn/Bakgrunn + reversering] \hfill \\
    Ser ikke noe konkret på dette. 
\end{description}

\subsubsection{Jacob Nilesens liste}

\begin{description}
  \item[1. Visibility of system status] \hfill \\
    Det er begrenset hvor mye status en bruker kan se på en nettside, men menylinja øverst viser til enhver tid hvilken fane brukeren er i.
  \item[2. Match between system and real world] \hfill \\
    Det brukes jevnt over et naturlig språk, som er lett å forstå. Dette er sett bort i fra eventuelle tekniske begreper, men disse blir ofte forklart i liten skrift. 
  \item[3. User control and freedom] \hfill \\
    Brukeren kan når som helst bevege seg dit han eller hun ønsker. 
  \item[4. Consistency and standards] \hfill \\
    Ser ingen tilfeller hvor dette ikke oppfylles. 
  \item[5. Error prevention] \hfill \\
   Ser ikke relevansen for denne på en internetframside. 
  \item[6. Recognintion reather than recall] \hfill \\
  Synes nettsiden gjør en god jobb her. Det er lett å forstå hva de ulike linkene/knappene gjør. Dette er også kanskje ikke veldig relevant på en nettside som dette, da det meste går i tekst og ikke ikoner.
  \item[7. Flexibility and effiency of use] \hfill \\
    Det er ikke noe særlig i nettsiden som øker effektiviteten for en ekspertbruker, men det er søkefelt tilgjengelig som kan øke effektiviteten litt. 
  \item[8. Aestetic and minimalist design] \hfill \\
  Alt er minimalt og det er ingen elementer som forstyrrer oppmersomheten til brukeren. 
  \item[9. Help users recognize, diagnose, and recover from errors] \hfill \\
    Lite relevant for en internettside, kanskje mer relevant for nettbutikken, men den er ikke endel av øvinga.   
  \item[10. Help and documentation] \hfill \\
   Lett å nå supportside.  
\end{description}

\subsubsection{Schneiderman}

\begin{description}
  \item[Strive for Consistency] \hfill \\
    Synes siden er veldig konsitent i designvalg. 
\item[Cater of universal usability] \hfill \\
  Det er mulig å benytte urlen som snarvei ved å skrive seg dit man vil, men utenom det finnes ingen tastaturkommandoer for snarveier.
  \item[Offer informative feedback] \hfill \\
    Linker i menyen endrer farge når musa panorerer over. Denne endringen av utseende gir en god feedback. 
  \item[Design dialogs to yield closure] \hfill \\
    Har ikke forsøkt et kjøp i nettbutikken, så kan ikke svare godt på dette spørsmålet.
  \item[Prevent errors] \hfill \\
    Vet ikke hvordan dette løses av nettsiden
  \item[Permit easy reversal of actions] \hfill \\
    Samme som forrige 
  \item[Support internal locus of control] \hfill \\
    Er ingen mulighet for å modifisere siden som man vil eller å logge inn på et personlig ``dashboard". 
  \item[Reduce short term memory] \hfill \\
    Det er lett å manuvrere frem og tilbake, så dette skulle ikke være et problem. 
\end{description}

\subsection{Delt opp etter brukergrupper?}

Forsiden er delt opp etter brukergrupper ved at man kan velge sitt ønskede produkt. Nettsiden er derfor ikke laget for en spesiell brukergruppe, men alle målgruppene.

\subsection{Hensikten med nettstedet}

Typisk brukssituasjon: 

\begin{itemize}
\item Finne informasjon om en ny datamaskin
\item Finne den siste nyheter fra produsenten
\end{itemize}

\subsubsection{Familiefaren vil ha ny datamaskin}

Øivind ønsker seg en datamaskin å bruke mens han pendler. Han går inn på nettsiden til Apple og trykker på ``Mac'' for å få oversikt over de tilgjengelige maskinene. Han er fornøyd med informasjonen og sorterer maskinene etter ytelse. 

\subsubsection{Ungdommen vil ha det nyeste}

Petter er ute etter det aller nyeste fra Apple, og vet at det snart kommer noe nytt. Derfor er han ofte inne på forsiden og sjekker hovednyheten midt på siden. Når nyheten er der, trykker han på bildet/annonsen og får opp siden til produktet. 

\subsection{Språk/ordvalg}

Apple benytter seg av et nøye gjennomtenkt ordvalg. Det er benyttet korte fraser for å vekke interesse om produktet. Det vil ikke være noe umiddelbart problem for noen av målgruppene å forstå hva Apple prøver å si. Det er mulig å endre språk på nettsiden til ønsket språk. 

\subsection{Linker, knapper, menyer\dots}

Selve menybaren er ganske lettforsåelig og det er tyldig hva som er knapper. Nyheter kommer tydlig frem. Hover-effekten på menybaren gjør at knapper skilles tydelig fra andre elementer. 


\subsection{Totalvurdering av brukervennlighet}

Jeg vil si at denne siden har en veldig brukervennlig design som er vanskelig å misforstå. Den ivaretar mange designprinsipper. Det er ingen særlig forstyrrende elementer på siden som skaper frustrasjon. Den er veldig brukervennlig for de fleste målgrupper. Det kan hende at bestemoren vil synes siden er litt vanskelig, men det går mer på dataforståelse enn brukervennlighet. 